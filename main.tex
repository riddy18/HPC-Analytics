\documentclass[conference]{csce}

\usepackage[hmargin=.75in,vmargin=1in]{geometry}
\usepackage[american]{babel}
\usepackage[T1]{fontenc}
\usepackage{times}
\usepackage{caption}

%%% Class name, option, and packages above are mandatory for generating an appropriate format 
%%% suitable for the CSCE style. Therefore, do not make any changes unless you know 
%%% what you are doing.
%%% However, if you need to use the subfig package, you must call it BEFORE the caption package.
%%% (NOTE: the subfig package probably will work but has not been tested.)

%%% The csce.cls is derived (in a quite dirty and quick manner) from the IEEEtrans.cls.
%%% At least the following packages are incompatible with the csce.cls:
%%% <DO NOT USE THEM> setspace, titlesec, amsthm
%%% There may be more, so if you use a package that produces a lot of errors or weird results, 
%%% be advised to avoid that package.

%%% Below packages are recommended to use for better results and compatible with the csce.cls
\usepackage{textcomp}
\usepackage{epsfig,graphicx}
\usepackage{xcolor}
\usepackage{amsfonts,amsmath,amssymb}
\usepackage{fixltx2e} % Fixing numbering problem when using figure/table* https://www.overleaf.com/project/5ba29f4a5004e017c0c7448c
\usepackage{booktabs}

%%% Below packages are probably useful for some table-formatting purposes. Compatibility is not yet
%%% tested but probably fine.
%\usepackage{tabularx}
%\usepackage{tabulary}
https://www.overleaf.com/project/5ba29f4a5004e017c0c7448c
%%% Using the hyperref package is not really necessary for conference papers, but if your paper includes
%%% a lot of URLs, and you wish them to be line-breakable, it might be useful.  When you need to use the
%%% hyperref package, make sure you set <colorlinks option> = true and all link colors black as shown in
%%% the sample below (the sample calls the ifpdf package, too).
%\usepackage{ifpdf} 
%\ifpdf
%\usepackage[pdftex,naturalnames,breaklinks=true,colorlinks=true,linkcolor=black,citecolor=black,filecolor=black,menucolor=black,urlcolor=black]{hyperref}
%\else
%\usepackage[dvips,naturalnames,breaklinks=true]{hyperref}
%\fi

\columnsep 6mm  %%% DO NOT CHANGE THIS


\title{\bf Cloudbursting: Stay-Or-Go }     %%%% Replace with your title.

%%%% Replace the author and institution/affiliation names. 
%%%% Make sure the author names are boldface.
\author{
{\bfseries Author1$^1$, Author2$^2$, Author3$^2$, and Author4$^3$}\\
Department of Computer Science, Kansas State University\\ Manhattan, Kansas, USA\\
}

\begin{document}


\maketitle %%%% To set Title and Author names.


\begin{abstract}%%%% Replace with your abstract.
We explored techniques/models for estimating
probability for an High Performance Computing cluster job to fail 
based on the user submission information available at run time to 
the scheduler. Using naive Bayes logistic regression, we got high 
accuracy in predicting jobs failed. However, the sensitivity 
of the results were abysmal and this was suspected to be due to the 
imbalanced data set used for the regression analysis. 
Apparently, 90 something percent of jobs submitted to the cluster passed.

Our ultimate aim is to provide "cloud-bursting" suggestions to the 
user to either keep running their job on the cluster or migrate their
job to cloud services like AmazonCloud. This is important as we want to 
be minimize the effects of job killed as a result of underestimation of
computing resources needed to complete jobs.

\end{abstract}


\vspace{1em}
\noindent\textbf{Keywords:}
 {\small  Replace with keywords} %%%% Replace with your keywords

%%%%%%%%%%%%%%%%%%%%%%%%%%%%%%%%%%%%%%%%%%%%%%%%%%%%%%%%%%%%


\section{Introduction}
abcd













 
\section{Utility Function}
We will model the Utility Estimation Model for Beocat using an MDP. Important components include:

Agent, Environment, State of a job, Actions, Utility

\subsection{Stay-Or-Go}

bbbb

\subsection{Ground Features and Relevance} 

ccc

\subsection{Data Preparation and Feature Analysis}
ddd

\subsection{Feature Construction for User Modeling}
eee

\subsection{Machine Learning Implementation}
fff

\subsection*{Prediction Techniques}

ggg

\subsection*{Classification Techniques}

hhh

\section{Evaluation}
iii

\section{Conclusions and Future Work}

jjj  

\subsection{Findings and Conclusions}

kkk

\subsection{Current and Future Work}

lll

\section{Acknowledgements}

mmm

\begin{thebibliography}{1}
	\providecommand{\url}[1]{#1}
	\csname url@rmstyle\endcsname
	\providecommand{\newblock}{\relax}
	\providecommand{\bibinfo}[2]{#2}
	\providecommand\BIBentrySTDinterwordspacing{\spaceskip=0pt\relax}
	\providecommand\BIBentryALTinterwordstretchfactor{4}
	\providecommand\BIBentryALTinterwordspacing{\spaceskip=\fontdimen2\font plus
		\BIBentryALTinterwordstretchfactor\fontdimen3\font minus
		\fontdimen4\font\relax}
	\providecommand\BIBforeignlanguage[2]{{%
			\expandafter\ifx\csname l@#1\endcsname\relax
			\typeout{** WARNING: IEEEtran.bst: No hyphenation pattern has been}%
			\typeout{** loaded for the language `#1'. Using the pattern for}%
			\typeout{** the default language instead.}%
			\else
			\language=\csname l@#1\endcsname
			\fi
			#2}}
	
	\bibitem{IEEE:confwithpaper}
	 Wolfgang. Gentzsch, ``Sun grid engine: Towards creating a compute power grid,'' 
 	in \emph{Cluster Computing and the Grid, 2001. Proceedings. First IEEE/ACM International Symposium on}, pp. 35-36. IEEE, 2001.
	
	\bibitem{IEEE:confwithpaper}
	Yoo, Andy B., Morris A. Jette, and Mark Grondona. ``Slurm: Simple linux utility for resource management,'' in \emph{Workshop on Job Scheduling Strategies for Parallel Processing}, pp. 44-60. Springer, Berlin, Heidelberg, 2003.
	
	\bibitem{IEEE: confwithpaper}
	Matsunaga, Andr{\'e}a and Fortes, Jos{\'e} AB. ``On the use of machine learning to predict the time and resources consumed by applications,'' in \emph{Proceedings of the 2010 10th IEEE/ACM International Conference on Cluster, Cloud and Grid Computing}, pp. 495-504. IEEE Computer Society, 2010.
	
	\bibitem{IEEE: journal}
	Bugbee, B., and Phillips, C., Egan, H., and Elmore, R., Gruchalla, K., and Purkayastha, A. "Prediction and characterization of application power use in a high-performance computing environment.'' \emph{Statistical Analysis and Data Mining: The ASA Data Science Journal} vol. 10, pp. 155-165, 2017.
	
	\bibitem{IEEE: confwithpaper}
	Berral, Josep Ll and Goiri, {\'I}{\~n}igo and Nou, Ram{\'o}n and Juli{\`a}, Ferran and Guitart, Jordi and Gavald{\`a}, Ricard and Torres, Jordi. ``Towards energy-aware scheduling in data centers using machine learning,'' in \emph{Proceedings of the 1st International Conference on energy-Efficient Computing and Networking}, pp. 215-224. ACM, 2010.
	
	\bibitem{IEEE: journal}
	Chou, J. S., Chiu, C. K., Farfoura, M., and Al-Taharwa, ``Optimizing the prediction accuracy of concrete compressive strength based on a comparison of data-mining techniques,''  \emph{Journal of Computing in Civil Engineering}., vol. 25, pp. 242-253, 2000.
	
	\bibitem{IEEE: confwithpaper}
	Li, J., Ma, X., Singh, K., Schulz, M., de Supinski, B. R., and McKee, S. A. ``Machine learning based online performance prediction for runtime parallelization and task scheduling,'' in \emph{Performance Analysis of Systems and Software, 2009. ISPASS 2009. IEEE International Symposium on}, pp. 89-100. IEEE, 2009.
	
	\bibitem{IEEEE: confwithpaper}
	Rodrigues, E. R., Cunha, R. L., Netto, M. A., and Spriggs, M. ``Helping HPC users specify job memory requirements via machine learning,'' in \emph{Proceedings of the Third International Workshop on HPC User Support Tools}, pp. 6-13, IEEE Press, 2016.
	
	\bibitem{IEEE: journal}
	Kohavi, R., and John, G. H., ``Wrappers for Feature Subset Selection'', \emph{Artificial Intelligence}, vol. 97, no. 1, pp. 273-324, 1997.
 
	\bibitem{IEEE: journal}
	Hsu, W. H., Welge, M., Redman, T., and Clutter, D.  ``High-Performance Commercial Data Mining: A Multistrategy Machine Learning Application.'' \emph{Data Mining and Knowledge Discovery}, vol. 6, no. 4, pp. 361-391, 2002.
    
	\bibitem{IEEE: book}
	Howes, T., and Smith, M, \emph{LDAP: programming directory-enabled applications with lightweight directory access protocol}. Sams Publishing, 1997.	

	\bibitem{IEEE: journal}
	Webb, G. I., Pazzani, M. J., and Billsus, D, ``Machine learning for user modeling,'' \emph{User modeling and user-adapted interaction}, vol. 11, pp. 19-29, 2001.
	
	\bibitem{IEEE: conference}
	Yao, Y., Zhao, Y., Wang, J., and Han, S, ``A model of machine learning based on user preference of attributes''. \emph{International Conference on Rough Sets and Current Trends in Computing}, pp. 587-596. Springer, Berlin, Heidelberg, 2006.

	\bibitem{IEEE: chapter}
    Yang, M., Hsu, W. H., and Kallumadi, S. ``Predictive Analytics of Social Networks: A Survey of Tasks and Techniques.'' In Hsu, W. H., ed., \emph{Emerging Methods in Predictive Analytics: Risk Management and Decision-Making}, pp. 297-333. IGI Global, 2016.
    
    \bibitem{IEEE: tutorial}
    Beutel, A., Akoglu, L., and Faloutsos, C. ``Graph-Based User Behavior Modeling: From Prediction to Fraud Detection''. In Cao, L., Zhang, C., Joachims, T., Webb, G. I., Margineantu, D. D., Williams, G., eds. \emph{Proceedings of the 21th ACM SIGKDD International Conference on Knowledge Discovery and Data Mining (KDD 2015), Tutorial Program}, pp. 2309-2310, 2015.
    
    \bibitem{IEEE: journal}
    Adedoyin-Olowe, M., Gaber, M. M., and Stahl, F., ``A Survey of Data Mining Techniques for Social Media Analysis'', vol. 2014, pp. 1-25, 2014.
    
    \bibitem{IEEE: conference}
    Hsu, W. H., Lancaster, J. P., Paradesi, M. S. R., and Weninger, T. ``Structural Link Analysis from User Profiles and Friends Networks: A Feature Construction Approach'', \emph{Proceedings of the 1st International Conference on Weblogs and Social Media (ICWSM 2007)}, pp. 75-80, 2007.

\bibitem{IEEE: journal}
    Pedregosa, F., Varoquaux, G., Gramfort, A., Michel, V., Thirion, B., Grisel, O., Blondel, M., Prettenhofer, P., Weiss, R., Dubourg, V., Vanderplas, J., Passos, A., Cournapeau, D., Brucher, M., Perrot, M., and Duchesnay, E. ``Scikit-learn: Machine Learning in Python'' \emph{Journal of Machine Learning Research}, vol. 12, pp. 2825-2830, 2011.

	\bibitem{IEEE: book}
	Massaron, L., and Boschetti, A, \emph{Regression Analysis with Python}. Packt Publishing Ltd, 2016.

\end{thebibliography}

\end{document}
